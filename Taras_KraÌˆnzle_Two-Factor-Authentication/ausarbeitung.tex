%%%%%%%%%%%%%%%%%%%%%%%%%%%%%%%%%%%%%%%%%%%%%%%%%%%%%%%%%%%%%%%%%%%%%%%%%%%%%
%
% Vorlage für Seminararbeiten im Institut für Verteilte Systeme
% 
% HINWEISE
% 
%  1. Bei Nutzung für Seminarausarbeitungen darf insbesondere die Schriftart
%     und -größe nicht angepasst werden.
%  2. Die Vorlage unterstützt deutsche und englische Ausarbeitungen durch
%     Anpassung der babel Paketoptionen.
%  3. Folgende Angaben sollen angepasst werden:
%     - Titel der Arbeit
%     - Name und E-Mail-Adresse des Autors
%     - Titel des Seminars
%     - Semester
%  4. Die Vorlage sieht eine Lizensierung unter CC-BY-SA vor, die jedoch
%     nicht verpflichtend ist. Falls nicht gewünscht, bitte alle \thanks
%     Befehle auskommentieren.
%     Die gewählte Lizenz (CC-BY-SA) ist kompatibel mit einer möglichen
%     Veröffentlichung auf dem Volltextserver der Uni Ulm
%     (http://vts.uni-ulm.de).
%
%%%%%%%%%%%%%%%%%%%%%%%%%%%%%%%%%%%%%%%%%%%%%%%%%%%%%%%%%%%%%%%%%%%%%%%%%%%%%

% Based on the IEEE Journal style.
\documentclass[10pt,a4paper,compsoc]{IEEEtran}

\usepackage{graphicx}
\usepackage[cmex10]{amsmath}
\usepackage[ngerman,USenglish]{babel} 
\usepackage{url}
\usepackage{hyperref}
\usepackage[utf8]{inputenc}

\newcommand\IEEEfirstsection[1]{%
  \noindent\raisebox{2\baselineskip}[0pt][0pt]%
  {\parbox{\columnwidth}{#1%
  \global\everypar=\everypar}}%
  \vspace{-1\baselineskip}\vspace{-\parskip}\par
}

\newcommand\cclicense{{\normalfont\sffamily\bfseries CC-BY-SA}}
\IfFileExists{ccicons.sty}{%
\usepackage{ccicons}
\renewcommand\cclicense{\ccbysa}
}

\begin{document}

\title{Two Factor Authentication}

\author{%
\IEEEauthorblockN{Taras Kr"anzle}\\
\IEEEauthorblockA{\url{taras-1.kraenzlel@uni-ulm.de}}%
\thanks{%
\cclicense{}
Diese Arbeit steht unter einer Creative Commons Namensnennung - Weitergabe unter gleichen Bedingungen 3.0 Deutschland Lizenz.}%
\thanks{\url{http://creativecommons.org/licenses/by-sa/3.0/de/}}%
}

\IEEEpubid{\sffamily%
\makebox[\columnwidth]{\hfill Ausgew"ahlte Themen Verteilter Systeme}%
\makebox[\columnsep]{$\cdot$}%
\makebox[\columnwidth]{WS 2013/14,
Institut f"ur Verteilte Systeme, Universit"at Ulm\hfill}}

\IEEEtitleabstractindextext{%
\begin{abstract}
Verifying a person is an always present topic in our society in terms of information security. This paper deals with the problems of Single Factor Authentication, such as wrong choices for passwords,phishing,social engineering and evaluates methods for enhancing the security, both in theory, by describing the possible combinations of known Authentication methods,but also in practice with examples from leading companies in the world. A perspective for the future is posed to alert the awareness for insecure Authentication methods.
\end{abstract}%
}

% make the title area
\maketitle

\tableofcontents

\IEEEfirstsection{%

\section{Introduction}
\label{sec:introduction}
}


\IEEEPARstart{I}n den vergangenen Jahren \dots Lim~\cite{test} id est laborum.

\section{ Authentication for Users}

\subsection{User Authentication Factors}
\subsubsection{Something You Know}
\subsubsection{Something You Have}
\subsubsection{Something You Are}

\subsection{Single Factor Authentication}
\subsubsection{Weak Authentication}
\subsubsection{Problems}

\section{Two Factor Authentication}


\subsection{Via mobile phones}
\subsubsection{Google}
Yet this example is for Google.
\subsubsection{Github}
This one for Github.
\subsubsection{airbnb}
This one for airbnb.
\subsection{Via TAN}
In the bank area this is used.
\subsection{Via Yubikey?}
Maybe?
\subsection{Biometric}
Maybe?
\subsection{Evaluation}
\section{Summary}
Clearly, there are certain pros and cons for the use.
\section{Conclusion}
Finally, ... .
\bibliographystyle{IEEEtranS}
\bibliography{references}

\end{document}
